\documentclass{beamer}

\newcommand{\course}{CS 1331 Introduction to Object Oriented Programming}
\newcommand{\lesson}{Inheritance, Part 3 of 3}
\newcommand{\code}{http://www.cc.gatech.edu/~simpkins/teaching/gatech/cs1331/code}

\author[Chris Simpkins] 
{Christopher Simpkins \\\texttt{chris.simpkins@gatech.edu}}
\institute[Georgia Tech] % (optional, but mostly needed)

\date[CS 1331]{}
\subject{\lesson}
% This is only inserted into the PDF information catalog. Can be left
% out. 

% If you have a file called "university-logo-filename.xxx", where xxx
% is a graphic format that can be processed by latex or pdflatex,
% resp., then you can add a logo as follows:

% \pgfdeclareimage[width=0.6in]{coc-logo}{cc_2012_logo}
% \logo{\pgfuseimage{coc-logo}}

\mode<presentation>
{
  \usetheme{Berlin}
  \useoutertheme{infolines}

  % or ...

 \setbeamercovered{transparent}
  % or whatever (possibly just delete it)
}

\usepackage{hyperref}
\usepackage{fancybox}
\usepackage{listings}
\usepackage[abbr]{harvard}

\usepackage[english]{babel}
% or whatever

\usepackage[latin1]{inputenc}
% or whatever

\usepackage{times}
\usepackage[T1]{fontenc}
% Or whatever. Note that the encoding and the font should match. If T1
% does not look nice, try deleting the line with the fontenc.


\usepackage{listings}
 
% "define" Scala
\lstdefinelanguage{scala}{
  morekeywords={abstract,case,catch,class,def,%
    do,else,extends,false,final,finally,%
    for,if,implicit,import,match,mixin,%
    new,null,object,override,package,%
    private,protected,requires,return,sealed,%
    super,this,throw,trait,true,try,%
    type,val,var,while,with,yield},
  otherkeywords={=>,<-,<\%,<:,>:,\#,@},
  sensitive=true,
  morecomment=[l]{//},
  morecomment=[n]{/*}{*/},
  morestring=[b]",
  morestring=[b]',
  morestring=[b]""",
}

\usepackage{color}
\definecolor{dkgreen}{rgb}{0,0.6,0}
\definecolor{gray}{rgb}{0.5,0.5,0.5}
\definecolor{mauve}{rgb}{0.58,0,0.82}
 
% Default settings for code listings
\lstset{frame=tb,
  language=scala,
  aboveskip=3mm,
  belowskip=3mm,
  showstringspaces=false,
  columns=flexible,
  basicstyle={\scriptsize\ttfamily},
  numbers=none,
  numberstyle=\tiny\color{gray},
  keywordstyle=\color{blue},
  commentstyle=\color{dkgreen},
  stringstyle=\color{mauve},
  frame=single,
  breaklines=true,
  breakatwhitespace=true,
  keepspaces=true
  %tabsize=3
}


\title[\course] % (optional, use only with long
                                      % paper titles)
{\lesson}

\subtitle{}
%% {Include Only If Paper Has a Subtitle}


% Delete this, if you do not want the table of contents to pop up at
% the beginning of each subsection:
%% \AtBeginSection[]
%% {
%%   \begin{frame}<beamer>{Outline}

%%  \tableofcontents[currentsection]
%%   \end{frame}
%% }

% If you wish to uncover everything in a step-wise fashion, uncomment
% the following command: 

% \beamerdefaultoverlayspecification{<+->}


\begin{document}

\begin{frame}
  \titlepage
\end{frame}


%------------------------------------------------------------------------
\begin{frame}[fragile]{A {\tt SalariedEmployee} Class}


Let's add {\tt SalariedEmployee} to our class hierarchy.  Here are the important pieces:
\begin{lstlisting}[language=Java]
public final class SalariedEmployee extends Employee {

    private static final int MONTHS_PER_YEAR = 12;
    private final double annualSalary;

    public SalariedEmployee(String aName, Date aHireDate,
                            double anAnnualSalary) {
        super(aName, aHireDate);
        disallowZeroesAndNegatives(anAnnualSalary);
        annualSalary = anAnnualSalary;
    }
    public double getAnnualSalary() {
        return annualSalary;
    }
    public double monthlyPay() {
        return annualSalary / MONTHS_PER_YEAR;
    }
    // ...
}
\end{lstlisting}

\end{frame}
%------------------------------------------------------------------------

%------------------------------------------------------------------------
\begin{frame}[fragile]{Our {\tt Employee} Class Hierarchy}


We now have all the classes in our hierarchy:
\vspace{-.1in}
\begin{center}
\includegraphics[height=1.2in]{employee-class-hierarchy.pdf}
\end{center}
\vspace{-.2in}
But our classes aren't well factored.
\begin{itemize}
\item {\tt SalariedEmployee} and {\tt HourlyEmployee} have duplicate copies of {\tt disallowZeroesAndNegatives}
\item {\tt SalariedEmployee} and {\tt HourlyEmployee} both have {\tt monthlyPay} methods, but these methods are not polymorphic because they're not defined in {\tt Employee} (you'll see what that means in the next lecture)
\end{itemize}

Over the next two lectures we'll refactor our {\tt Employee} class hierarchy to give it a clean object-oriented design and learn a few important details about fitting into the Java class hierarchy rooted at {\tt java.lang.Object}.

\end{frame}
%------------------------------------------------------------------------

%------------------------------------------------------------------------
\begin{frame}[fragile]{Refactoring Common Code Into a Superclass}


Let's move the definition of {\tt disallowZeroesAndNegatives} into {\tt Employee} so it will be shared (rather than duplicated) in {\tt SalariedEmployee} and {\tt HourlyEmployee}.\\
\vspace{.05in}
After cutting {\tt disallowZeroesAndNegatives} from {\tt SalariedEmployee} and {\tt HourlyEmployee} and pasting it into {\tt Employee}, {\tt javac} tells us:
\vspace{-.1in}
\begin{lstlisting}[language=Java]
$ javac Employee.java HourlyEmployee.java SalariedEmployee.java
HourlyEmployee.java:25: cannot find symbol
symbol  : method disallowZeroesAndNegatives(double,double)
location: class HourlyEmployee
        disallowZeroesAndNegatives(anHourlyWage, aMonthlyHours);
        ^
SalariedEmployee.java:17: cannot find symbol
symbol  : method disallowZeroesAndNegatives(double)
location: class SalariedEmployee
        disallowZeroesAndNegatives(anAnnualSalary);
        ^
2 errors
\end{lstlisting}
% $
\vspace{-.1in}
Why did we get these errors?

\end{frame}
%------------------------------------------------------------------------

%------------------------------------------------------------------------
\begin{frame}[fragile]{{\tt protected} Members}


{\tt private} members of a superclass are effectively not inherited by subclasses.  To make a member accessible to subclasses, use {\tt protected}:
\begin{lstlisting}[language=Java]
public class Employee {
    protected void disallowZeroesAndNegatives(double ... args) {
        // ...
    }
    // ...
}
\end{lstlisting}
{\tt protected} members
\begin{itemize}
\item are accessible to subclasses and other classes in the same package, and
\item can be overriden in subclasses.
\end{itemize}
{\tt protected} members provide encapsulation within a class hierarchy, {\tt private} provides encapsulation within a single class.

\end{frame}
%------------------------------------------------------------------------

%------------------------------------------------------------------------
\begin{frame}[fragile]{Overriding verus Overloading}


\begin{lstlisting}[language=Java]

\end{lstlisting}

\begin{itemize}
\item
\end{itemize}


\end{frame}
%------------------------------------------------------------------------


%------------------------------------------------------------------------
\begin{frame}[fragile]{Covariant Returns In Overridden Methods}


\begin{lstlisting}[language=Java]

\end{lstlisting}

\begin{itemize}
\item
\end{itemize}


\end{frame}
%------------------------------------------------------------------------

%------------------------------------------------------------------------
\begin{frame}[fragile]{Visibility Modification in Overriden Methods}


\begin{lstlisting}[language=Java]

\end{lstlisting}

\begin{itemize}
\item
\end{itemize}


\end{frame}
%------------------------------------------------------------------------

%------------------------------------------------------------------------
\begin{frame}[fragile]{}


{\tt java.lang.Object} defines several methods that are designed to be overriden in subclasses: 
\begin{itemize}
\item {\tt equals}
\item {\tt hashCode}
\item {\tt toString}
\item {\tt clone}
\item {\tt finalize}.
\end{itemize}


\end{frame}
%------------------------------------------------------------------------


%------------------------------------------------------------------------
\begin{frame}[fragile]{When to Override the {\tt equals} Method}


The default implementation of {\tt equals} in {\tt java.lang.Object} is object identity - each object {\tt equals} only itself.

When should a class override {\tt equals}?
\begin{itemize}
\item When logical equality differes from object identity, as is the case for {\it value} classes like {\tt Date}
\item When classes that don't implement instance control.
  \begin{itemize}
  \item Instance control means that a class ensures that there is only one instance of a class.  
  \end{itemize}
\item When a suitable {\tt equals} method is not provided by a superclass.
\end{itemize}

More important than recognizing {\it when} to override {\tt equals} is knowing {\tt how} to override {\tt equals}.
\end{frame}
%------------------------------------------------------------------------

%------------------------------------------------------------------------
\begin{frame}[fragile]{How to Override the {\tt equals} Method}


Obey the general contract of {\tt equals} (\href{}{JLS }), which says that {\tt equals} defines an equivalence relation.  So, for non-null references, {\tt equals} is
\begin{itemize}
\item reflexive - any object {\tt equals} itself
\item symmetric - if {\tt a.equals(b)} is true then {\tt b.equals(a)} must be true
\item transitive - if {\tt a.equals(b)} and {\tt b.equals(c)} is true then {\tt a.equals(c)} must be true
\item consistent - if {\tt a} and {\tt b} do not change between invocations of {\tt a.equals(b)} or {\tt b.equals(a)} then each invocation must return the same result
\item a.equals(null) must always return {\tt false}.
\end{itemize}


\end{frame}
%------------------------------------------------------------------------


%------------------------------------------------------------------------
\begin{frame}[fragile]{}


\begin{lstlisting}[language=Java]

\end{lstlisting}
General recipe for proper equals method
\begin{enumerate}
\item check this == that
\item check that instanceof this
\item cast that to this.class (guaranteed to work after instanceof test)
\item check equals on each ``significant'' field
\end{enumerate}


\end{frame}
%------------------------------------------------------------------------




% %------------------------------------------------------------------------
% \begin{frame}[fragile]{}


% \begin{lstlisting}[language=Java]

% \end{lstlisting}

% \begin{itemize}
% \item
% \end{itemize}


% \end{frame}
% %------------------------------------------------------------------------


\end{document}
