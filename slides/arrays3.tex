\documentclass[xcolor={table}]{beamer}

\newcommand{\course}{CS 1331 Introduction to Object Oriented Programming}
\newcommand{\lesson}{Arrays, Part 3 of 3}
\newcommand{\code}{http://www.cc.gatech.edu/~simpkins/teaching/gatech/cs1331/code}

\author[Chris Simpkins] 
{Christopher Simpkins \\\texttt{chris.simpkins@gatech.edu}}
\institute[Georgia Tech] % (optional, but mostly needed)

\date[CS 1331]{}
\subject{\lesson}
% This is only inserted into the PDF information catalog. Can be left
% out. 

% If you have a file called "university-logo-filename.xxx", where xxx
% is a graphic format that can be processed by latex or pdflatex,
% resp., then you can add a logo as follows:

% \pgfdeclareimage[width=0.6in]{coc-logo}{cc_2012_logo}
% \logo{\pgfuseimage{coc-logo}}

\mode<presentation>
{
  \usetheme{Berlin}
  \useoutertheme{infolines}

  % or ...

 \setbeamercovered{transparent}
  % or whatever (possibly just delete it)
}


\usepackage{hyperref}
\usepackage{fancybox}
\usepackage{listings}
\usepackage[abbr]{harvard}

\usepackage[english]{babel}
% or whatever

\usepackage[latin1]{inputenc}
% or whatever

\usepackage{times}
\usepackage[T1]{fontenc}
% Or whatever. Note that the encoding and the font should match. If T1
% does not look nice, try deleting the line with the fontenc.


\usepackage{listings}
 
% "define" Scala
\lstdefinelanguage{scala}{
  morekeywords={abstract,case,catch,class,def,%
    do,else,extends,false,final,finally,%
    for,if,implicit,import,match,mixin,%
    new,null,object,override,package,%
    private,protected,requires,return,sealed,%
    super,this,throw,trait,true,try,%
    type,val,var,while,with,yield},
  otherkeywords={=>,<-,<\%,<:,>:,\#,@},
  sensitive=true,
  morecomment=[l]{//},
  morecomment=[n]{/*}{*/},
  morestring=[b]",
  morestring=[b]',
  morestring=[b]""",
}

\usepackage{color}
\definecolor{dkgreen}{rgb}{0,0.6,0}
\definecolor{gray}{rgb}{0.5,0.5,0.5}
\definecolor{mauve}{rgb}{0.58,0,0.82}
 
% Default settings for code listings
\lstset{frame=tb,
  language=scala,
  aboveskip=3mm,
  belowskip=3mm,
  showstringspaces=false,
  columns=flexible,
  basicstyle={\scriptsize\ttfamily},
  numbers=none,
  numberstyle=\tiny\color{gray},
  keywordstyle=\color{blue},
  commentstyle=\color{dkgreen},
  stringstyle=\color{mauve},
  frame=single,
  breaklines=true,
  breakatwhitespace=true,
  keepspaces=true
  %tabsize=3
}


\title[\course] % (optional, use only with long
                                      % paper titles)
{\lesson}

\subtitle{}
%% {Include Only If Paper Has a Subtitle}


% Delete this, if you do not want the table of contents to pop up at
% the beginning of each subsection:
%% \AtBeginSection[]
%% {
%%   \begin{frame}<beamer>{Outline}

%%  \tableofcontents[currentsection]
%%   \end{frame}
%% }

% If you wish to uncover everything in a step-wise fashion, uncomment
% the following command: 

% \beamerdefaultoverlayspecification{<+->}


\begin{document}

\begin{frame}
  \titlepage
\end{frame}


%------------------------------------------------------------------------
\begin{frame}[fragile]{Multidimensional Arrays}


You can create arrays of any number of dimensions simply by adding additional square-bracketed sizes.  For example:

\begin{lstlisting}[language=Java]
char[][] grid;
\end{lstlisting}
The declaration statement above:
\begin{itemize}
\item Declares a 2-dimensional array of  {\tt char}.
\item As with one-dimensinal arrays, {\tt char} is the base type.
\item Each element of {\tt grid}, which is indexed by two {\tt int} expressions, is a {\tt char} variable.
\end{itemize}


\end{frame}
%------------------------------------------------------------------------

%------------------------------------------------------------------------
\begin{frame}[fragile]{Initializing Multidimensional Arrays}


Initialization of 2-dimensional arrays can be done with {\tt new}:
\begin{lstlisting}[language=Java]
grid = new char[10][10];
\end{lstlisting}

or with literal initialization syntax:
\begin{lstlisting}[language=Java]
char[][] grid = {{' ', ' ', ' ', ' ', ' ', ' ', ' ', ' ', ' ', ' '},
                 {' ', ' ', ' ', ' ', ' ', ' ', ' ', ' ', ' ', ' '},
                 {' ', '*', '*', ' ', ' ', ' ', ' ', '*', '*', ' '},
                 {' ', '*', '*', ' ', ' ', ' ', ' ', '*', '*', ' '},
                 {' ', ' ', ' ', ' ', '*', '*', ' ', ' ', ' ', ' '},
                 {' ', ' ', ' ', ' ', '*', '*', ' ', ' ', ' ', ' '},
                 {' ', '*', ' ', ' ', ' ', ' ', ' ', ' ', '*', ' '},
                 {' ', ' ', '*', ' ', ' ', ' ', ' ', '*', ' ', ' '},
                 {' ', ' ', ' ', '*', '*', '*', '*', ' ', ' ', ' '},
                 {' ', ' ', ' ', ' ', ' ', ' ', ' ', ' ', ' ', ' '}};
\end{lstlisting}

Notice that a 2-dimensional array is an array of 1-dimensional arrays (and a 3-dimensional array is an array of 2-dimensional arrays, and so on).

\end{frame}
%------------------------------------------------------------------------

%------------------------------------------------------------------------
\begin{frame}[fragile]{Visualizing Multidimensional Arrays}


Our 2-dimensional {\tt grid} array can be visualized as a 2-d grid of cells.\\
\vspace{.1in}
\begin{tabular}{p{.4in}p{.2in}p{.2in}p{.2in}p{.2in}p{.2in}p{.2in}p{.2in}p{.2in}p{.2in}p{.2in}}
         & [0] & [1] & [2] & [3] & [4] & [5] & [6] & [7] & [8] & [9]
\end{tabular}
\begin{tabular}{p{.4in}|p{.2in}|p{.2in}|p{.2in}|p{.2in}|p{.2in}|p{.2in}|p{.2in}|p{.2in}|p{.2in}|p{.2in}|}\cline{2-11}
grid[0] & ' ' & ' ' & ' ' & ' ' & ' ' & ' ' & ' ' & ' ' & ' ' & ' ' \\
\cline{2-11}
grid[1] & ' ' & ' ' & ' ' & ' ' & ' ' & ' ' & ' ' & ' ' & ' ' & ' ' \\
\cline{2-11}
grid[2] & ' ' & '*' & '*' & ' ' & ' ' & ' ' & ' ' & '*' & '*' & ' ' \\
\cline{2-11}
grid[3] & ' ' & '*' & \cellcolor{yellow}'*' & ' ' & ' ' & ' ' & ' ' & '*' & '*' & ' ' \\
\cline{2-11}
grid[4] & ' ' & ' ' & ' ' & ' ' & '*' & '*' & ' ' & ' ' & ' ' & ' ' \\
\cline{2-11}
grid[5] & ' ' & ' ' & ' ' & ' ' & '*' & '*' & ' ' & ' ' & ' ' & ' ' \\
\cline{2-11}
grid[6] & ' ' & '*' & ' ' & ' ' & ' ' & ' ' & ' ' & ' ' & '*' & ' ' \\
\cline{2-11}
grid[7] & ' ' & ' ' & '*' & ' ' & ' ' & ' ' & ' ' & '*' & ' ' & ' ' \\
\cline{2-11}
grid[8] & ' ' & ' ' & ' ' & '*' & '*' & '*' & '*' & ' ' & ' ' & ' ' \\
\cline{2-11}
grid[9] & ' ' & ' ' & ' ' & ' ' & ' ' & ' ' & ' ' & ' ' & ' ' & ' ' \\
\cline{2-11}
\end{tabular}\\
\vspace{.1in}
And an individual cell can be accessed by supplying two indices:

\begin{lstlisting}[language=Java]
grid[3][2] == '*'; // true
\end{lstlisting}

\end{frame}
%------------------------------------------------------------------------

%------------------------------------------------------------------------
\begin{frame}[fragile]{Traversing Multidimensional Arrays}


Traverse 2-dimensional array by nesting loops.  The key to getting it right is to use the right {\tt length}s.
\begin{lstlisting}[language=Java]
for (int row = 0; row < grid.length; ++row) {
    for (int col = 0; col < grid[row].length; ++col) {
        System.out.print(grid[row][col]);
    }
    System.out.println();
}
\end{lstlisting}
Note that the for loops above traverse the grid in row-major order.  We can traverse the grid in column-major order by reversing the nesting of the for-loops:
\begin{lstlisting}[language=Java]
for (int col = 0; col < grid[0].length; ++col) {
    for (int row = 0; row < grid.length; ++row) {
        System.out.print(grid[row][col]);
    }
    System.out.println();
}
\end{lstlisting}


\end{frame}
%------------------------------------------------------------------------


%------------------------------------------------------------------------
\begin{frame}[fragile]{Ragged Arrays}


It's possible to create {\it ragged arrays} by creating nested arrays of variable length.  For example:
\begin{lstlisting}[language=Java]
double [][] ragged = new double[3][];
ragged[0] = new double[5];
ragged[1] = new double[10];
ragged[2] = new double[4];
\end{lstlisting}

Can we traverse array {\tt ragged} in row-major order?

Can we traverse array {\tt ragged} in column-major order?

\end{frame}
%------------------------------------------------------------------------

%------------------------------------------------------------------------
\begin{frame}[fragile]{Programming Exercise}

\begin{itemize}
\item Download \href{\code/array-data.csv}{array-data.csv}.
\item Let's write a program to read the data from \href{\code/array-data.csv}{array-data.csv} into an array.
\end{itemize}

\end{frame}
%------------------------------------------------------------------------


% %------------------------------------------------------------------------
% \begin{frame}[fragile]{}


% \begin{lstlisting}[language=Java]

% \end{lstlisting}

% \begin{itemize}
% \item
% \end{itemize}


% \end{frame}
% %------------------------------------------------------------------------


\end{document}
