\documentclass{beamer}

\newcommand{\course}{CS 1331 Introduction to Object Oriented Programming}
\newcommand{\lesson}{Loops}
\newcommand{\code}{http://www.cc.gatech.edu/~simpkins/teaching/gatech/cs1331/code}

\author[Chris Simpkins] 
{Christopher Simpkins \\\texttt{chris.simpkins@gatech.edu}}
\institute[Georgia Tech] % (optional, but mostly needed)

\date{}

\subject{\lesson}


% If you have a file called "university-logo-filename.xxx", where xxx
% is a graphic format that can be processed by latex or pdflatex,
% resp., then you can add a logo as follows:

% \pgfdeclareimage[width=0.6in]{coc-logo}{cc_2012_logo}
% \logo{\pgfuseimage{coc-logo}}

\mode<presentation>
{
  \usetheme{Berlin}
  \useoutertheme{infolines}

  % or ...

 \setbeamercovered{transparent}
  % or whatever (possibly just delete it)
}

\usepackage{hyperref}
\usepackage{fancybox}
\usepackage{listings}
\usepackage[abbr]{harvard}

\usepackage[english]{babel}
% or whatever

\usepackage[utf8]{inputenc}
% or whatever

\usepackage{times}
\usepackage[T1]{fontenc}
% Or whatever. Note that the encoding and the font should match. If T1
% does not look nice, try deleting the line with the fontenc.


\usepackage{listings}
 
% "define" Scala
\lstdefinelanguage{scala}{
  morekeywords={abstract,case,catch,class,def,%
    do,else,extends,false,final,finally,%
    for,if,implicit,import,match,mixin,%
    new,null,object,override,package,%
    private,protected,requires,return,sealed,%
    super,this,throw,trait,true,try,%
    type,val,var,while,with,yield},
  otherkeywords={=>,<-,<\%,<:,>:,\#,@},
  sensitive=true,
  morecomment=[l]{//},
  morecomment=[n]{/*}{*/},
  morestring=[b]",
  morestring=[b]',
  morestring=[b]""",
}

\usepackage{color}
\definecolor{dkgreen}{rgb}{0,0.6,0}
\definecolor{gray}{rgb}{0.5,0.5,0.5}
\definecolor{mauve}{rgb}{0.58,0,0.82}
 
% Default settings for code listings
\lstset{frame=tb,
  language=scala,
  aboveskip=2mm,
  belowskip=2mm,
  showstringspaces=false,
  columns=flexible,
  basicstyle={\scriptsize\ttfamily},
  numbers=none,
  numberstyle=\tiny\color{gray},
  keywordstyle=\color{blue},
  commentstyle=\color{dkgreen},
  stringstyle=\color{mauve},
  frame=single,
  breaklines=true,
  breakatwhitespace=true,
  keepspaces=true
  %tabsize=3
}


\title[\course] % (optional, use only with long
                                      % paper titles)
{\lesson}

\subtitle{}
%% {Include Only If Paper Has a Subtitle}

\newcommand{\link}[2]{\href{#1}{\textcolor{blue}{\underline{#2}}}}


% \beamerdefaultoverlayspecification{<+->}


\begin{document}

\begin{frame}
  \titlepage
\end{frame}



%------------------------------------------------------------------------
\begin{frame}[fragile]{Loops and Iteration}


Algorithms often call for repeated action or iteration, e.g. :
\begin{itemize}
\item ``repeat ... while (or until) some condition is true'' (looping) or 
\item ``for each element of this array/list/etc. ...'' (iteration)
\end{itemize}

Java provides three iteration structures:
\begin{itemize}
\item {\tt while} loop
\item {\tt do-while} loop
\item {\tt for} iteration statement
\end{itemize}


\end{frame}
%------------------------------------------------------------------------

%------------------------------------------------------------------------
\begin{frame}[fragile]{{\tt while} and {\tt do-while}}

{\tt while} loops are pre-test loops: the loop condition is tested before the loop body is executed
\begin{lstlisting}[language=Java]
while (condition) { // condition is any boolean expression
      // loop body executes as long as condition is true
}
\end{lstlisting}

{\tt do-while} loops are post-test loops: the loop condition is tested after the loop body is executed
\begin{lstlisting}[language=Java]
do {
      // loop body executes as long as condition is true
} while (condition)
\end{lstlisting}
The body of a {\tt do-while} loop will always execute at least once.

\end{frame}
%------------------------------------------------------------------------

%------------------------------------------------------------------------
\begin{frame}[fragile]{{\tt for} Statements}


The general {\tt for} statement syntax is:
\begin{lstlisting}[language=Java]
for(initializer; condition; update) {
     // body executed as long as condition is true
}
\end{lstlisting}
\begin{itemize}
\item {\tt intializer} is an assignment statement that sets the initial value of the loop variables also called the loop index
\item {\tt condition} is tested at the beginning of the loop body to determine whether the loop body should be executed.  When false, the loop exits just like a while loop
\item {\tt update} is an expression that updates the loop variables
\end{itemize}

\end{frame}
%------------------------------------------------------------------------

%------------------------------------------------------------------------
\begin{frame}[fragile]{{\tt for} Statement vs. {\tt while} Statement}


The {\tt for} statement:
\begin{lstlisting}[language=Java]
for(initializer; condition; update) {
     // body executed as long as condition is true
}
\end{lstlisting}

is equivalent to:
\begin{lstlisting}[language=Java]
initializer;
while (condition) {
  // body
  update;
}
\end{lstlisting}


{\tt for} is Java's primary iteration structure.  In the future we'll see generalized versions, but for now {\tt for} statements are used primarily to iterate through the indexes of data structures and to repeat code a particular number of times.


\end{frame}
%------------------------------------------------------------------------

%------------------------------------------------------------------------
\begin{frame}[fragile]{{\tt for} Statement Examples}


Here's an example adapted from  \link{\code/Switch.java}{Switch.java}.  We use the {\tt for} loops index variable to visit each character in a {\tt String} and count the digits and letters:
\begin{lstlisting}[language=Java]
int digitCount = 0, letterCount = 0;
for (int i = 0; i < input.length(); ++i) {
    char c = input.charAt(i);
    if (Character.isDigit(c)) digitCount++;
    if (Character.isAlphabetic(c)) letterCount++;
}
\end{lstlisting}

And here's a simple example of repeating an action a fixed number of times:
\begin{lstlisting}[language=Java]
for (int i = 0; i < 10; ++i)
        System.out.println("Meow!");
\end{lstlisting}



\end{frame}
%------------------------------------------------------------------------

%------------------------------------------------------------------------
\begin{frame}[fragile]{{\tt for} Statement Subtleties}


Better to declare loop index in {\tt for} to limit it's scope.  Prefer:
\vspace{-.05in}
\begin{lstlisting}[language=Java]
for (int i = 0; i < 10; ++i)
\end{lstlisting}
\vspace{-.1in}
to:
\vspace{-.05in}
\begin{lstlisting}[language=Java]
int i; // Bad.  Looop index variable visible outside loop.
for (i = 0; i < 10; ++i)
\end{lstlisting}
\vspace{-.08in}
You can have multiple loop indexes separated by commas:
\vspace{-.05in}
\begin{lstlisting}[language=Java]
String mystery = "mnerigpaba", solved = ""; int len = mystery.length();
for (int i = 0, j = len - 1; i < len/2; ++i, --j) {
    solved = solved + mystery.charAt(i) + mystery.charAt(j);
}
\end{lstlisting}
\vspace{-.1in}
Note that the loop above is one loop, not nested loops.\\
\vspace{.025in}
Beware of common ``extra semicolon'' syntax error:
\vspace{-.08in}
\begin{lstlisting}[language=Java]
for (int i = 0; i < 10; ++i); // oops!  semicolon ends the statement
    print(meow);  // this will only execute once, not 10 times
\end{lstlisting}

\end{frame}
%------------------------------------------------------------------------

%------------------------------------------------------------------------
\begin{frame}[fragile]{Forever is a Long Time}


Here are two idioms for indefinite loops.  First with {\tt for}:
\begin{lstlisting}[language=Java]
for (;;) {
    // ever
}
\end{lstlisting}

and with {\tt while}:
\begin{lstlisting}[language=Java]
while (true) {
    // forever
}
\end{lstlisting}

Let's play with some loop examples in \link{\code/Loops.java}{Loops.java}.

\end{frame}
%------------------------------------------------------------------------


%------------------------------------------------------------------------
\begin{frame}[fragile]{{\tt break} and {\tt continue}}


Java provides two non-structured-programming ways to alter loop control flow:
\begin{itemize}
\item {\tt break} exits the loop, possibly to a labeled location in the program
\item {\tt continue} skips the remainder of a loop body and continues with the next iteration
\end{itemize}

Consider the following while loop:
\begin{lstlisting}[language=Java]
boolean shouldContinue = true;
while (shouldContinue) {
    System.out.println("Enter some input (exit to quit):");
    String input = kbd.next();
    doSomethingWithInput(input); // We do something with "exit" too.
    keepGoing = (input.equalsIgnoreCase("exit")) ? false : true;
}
\end{lstlisting}
We don't test for the termination sentinal, ``exit,'' until after we do something with it.  Situations like these often tempt us to use {\tt break} ...

\end{frame}
%------------------------------------------------------------------------

%------------------------------------------------------------------------
\begin{frame}[fragile]{{\tt break}ing out of a {\tt while} Loop}


We could test the sentinal before processing and {\tt break}:
\vspace{-.05in}
\begin{lstlisting}[language=Java]
boolean shouldContinue = true;
while (shouldContinue) {
    System.out.println("Enter some input (exit to quit):");
    String input = kbd.next();
    if (input.equalsIgnoreCase("exit")) break;
    doSomethingWithInput(input);
}
\end{lstlisting}
\vspace{-.1in}
But it's better to use structured programming:
\vspace{-.05in}
\begin{lstlisting}[language=Java]
boolean shouldContinue = true;
while (shouldContinue) {
    System.out.println("Enter some input (exit to quit):");
    String input = kbd.next();
    if (input.equalsIgnoreCase("exit")) {
        shouldContinue = false;
    } else {
        doSomethingWithInput(input);
    }
}
\end{lstlisting}


\end{frame}
%------------------------------------------------------------------------


%------------------------------------------------------------------------
\begin{frame}[fragile]{Closing Thoughts}


\begin{itemize}
\item We now have all the tools we need to implement any algorithm:
\begin{itemize}
\item Sequence
\item Selection
\item Repetition
\end{itemize}
\item Now we can start learning about Java's support for OOP.
\end{itemize}


\end{frame}
%------------------------------------------------------------------------


\end{document}
