\documentclass{beamer}

\newcommand{\course}{CS 1331 Introduction to Object Oriented Programming}
\newcommand{\lesson}{Inheritance, Part 2 of 2}
\newcommand{\code}{http://www.cc.gatech.edu/~simpkins/teaching/gatech/cs1331/code}

\author[Chris Simpkins] 
{Christopher Simpkins \\\texttt{chris.simpkins@gatech.edu}}
\institute[Georgia Tech] % (optional, but mostly needed)

\date[CS 1331]{}

\subject{\lesson}


% If you have a file called "university-logo-filename.xxx", where xxx
% is a graphic format that can be processed by latex or pdflatex,
% resp., then you can add a logo as follows:

% \pgfdeclareimage[width=0.6in]{coc-logo}{cc_2012_logo}
% \logo{\pgfuseimage{coc-logo}}

\mode<presentation>
{
  \usetheme{Berlin}
  \useoutertheme{infolines}

  % or ...

 \setbeamercovered{transparent}
  % or whatever (possibly just delete it)
}

\usepackage{hyperref}
\usepackage{fancybox}
\usepackage{listings}
\usepackage[abbr]{harvard}

\usepackage[english]{babel}
% or whatever

\usepackage[utf8]{inputenc}
% or whatever

\usepackage{times}
\usepackage[T1]{fontenc}
% Or whatever. Note that the encoding and the font should match. If T1
% does not look nice, try deleting the line with the fontenc.


\usepackage{listings}
 
% "define" Scala
\lstdefinelanguage{scala}{
  morekeywords={abstract,case,catch,class,def,%
    do,else,extends,false,final,finally,%
    for,if,implicit,import,match,mixin,%
    new,null,object,override,package,%
    private,protected,requires,return,sealed,%
    super,this,throw,trait,true,try,%
    type,val,var,while,with,yield},
  otherkeywords={=>,<-,<\%,<:,>:,\#,@},
  sensitive=true,
  morecomment=[l]{//},
  morecomment=[n]{/*}{*/},
  morestring=[b]",
  morestring=[b]',
  morestring=[b]""",
}

\usepackage{color}
\definecolor{dkgreen}{rgb}{0,0.6,0}
\definecolor{gray}{rgb}{0.5,0.5,0.5}
\definecolor{mauve}{rgb}{0.58,0,0.82}
 
% Default settings for code listings
\lstset{frame=tb,
  language=scala,
  aboveskip=2mm,
  belowskip=2mm,
  showstringspaces=false,
  columns=flexible,
  basicstyle={\scriptsize\ttfamily},
  numbers=none,
  numberstyle=\tiny\color{gray},
  keywordstyle=\color{blue},
  commentstyle=\color{dkgreen},
  stringstyle=\color{mauve},
  frame=single,
  breaklines=true,
  breakatwhitespace=true,
  keepspaces=true
  %tabsize=3
}


\title[\course] % (optional, use only with long
                                      % paper titles)
{\lesson}

\subtitle{}
%% {Include Only If Paper Has a Subtitle}

\newcommand{\link}[2]{\href{#1}{\textcolor{blue}{\underline{#2}}}}


% \beamerdefaultoverlayspecification{<+->}


\begin{document}

\begin{frame}
  \titlepage
\end{frame}

%------------------------------------------------------------------------
\begin{frame}[fragile]{Access Modifiers}


\begin{center}
\begin{tabular}{|l|c|c|c|c|} \hline
Modifier & Class & Package & Subclass & World\\
\hline
public & Y & Y & Y & Y\\
protected & Y & Y & Y & N\\
no modifier & Y & Y & N & N\\
private & Y & N & N & N\\
\hline
\end{tabular}
\end{center}

\begin{itemize}
\item Every class has an access level (for now all of our classes are {\tt public}).
\item Every member has an access level.
\end{itemize}

\end{frame}
%------------------------------------------------------------------------

%------------------------------------------------------------------------
\begin{frame}[fragile]{Copy Constructors}


A copy constructor simplifies new object construction.  Here are the copy constructors for {\tt Emplyoyee}
\vspace{-.05in}
\begin{lstlisting}[language=Java]
public class Employee {
    public Employee(Employee other) {
        this.name = other.name;
        this.hireDate = other.hireDate;
    } // ...
}
\end{lstlisting}

and {\tt HourlyEmployee}
\vspace{-.05in}
\begin{lstlisting}[language=Java]
public class HourlyEmployee extends Employee {
   public HourlyEmployee(HourlyEmployee other) {
        super(other);
        this.hourlyWage = other.hourlyWage;
        this.monthlyHours = other.monthlyHours;
    } // ...
}
\end{lstlisting}

Can you think of an important consideration in writing copy constructors?

\end{frame}
%------------------------------------------------------------------------


%------------------------------------------------------------------------
\begin{frame}[fragile]{Explicit Constructor Invocation with {\tt this}}


What if we wanted to have default default values for hourly wages and monthly hours?  We can provide an alternate constructor that delegates to our main constructor with {\tt this}:
\begin{lstlisting}[language=Java]
public final class HourlyEmployee extends Employee {
    /**
     * Constructs an HourlyEmployee with hourly wage of 20 and 
     * monthly hours of 160.
     */
    public HourlyEmployee(String aName, Date aHireDate) {
        this(aName, aHireDate, 20.00, 160.0);
    }
    public HourlyEmployee(String aName, Date aHireDate,
                          double anHourlyWage, double aMonthlyHours) {
        super(aName, aHireDate);
        disallowZeroesAndNegatives(anHourlyWage, aMonthlyHours);
        hourlyWage = anHourlyWage;
        monthlyHours = aMonthlyHours;
    }
    // ...
}
\end{lstlisting}

\begin{itemize}
\item If present, an explicit constructor call must be the first statement in the constructor.
\item Can't have both a {\tt super} and {\tt this} call in a constructor.
\item A constructor with a {\tt this} call must call, either directly or indirectly, a constructor with a {\tt super} call (implicit or explicit).
\end{itemize}


\end{frame}
%------------------------------------------------------------------------

%------------------------------------------------------------------------
\begin{frame}[fragile]{{\tt this} and {\tt super}}


\begin{itemize}
\item If present, an explicit constructor call must be the first statement in the constructor.
\item Can't have both a {\tt super} and {\tt this} call in a constructor.
\item A constructor with a {\tt this} call must call, either directly or indirectly, a constructor with a {\tt super} call (implicit or explicit).
\end{itemize}

\begin{lstlisting}[language=Java]
public final class HourlyEmployee extends Employee {
    public HourlyEmployee(String aName, Date aHireDate) {
        this(aName, aHireDate, 20.00);
    }
    public HourlyEmployee(String aName, Date aHireDate, souble anHourlyWage) {
        this(aName, aHireDate, anHourlyWage, 160.0);
    }
    public HourlyEmployee(String aName, Date aHireDate,
                          double anHourlyWage, double aMonthlyHours) {
        super(aName, aHireDate);
        disallowZeroesAndNegatives(anHourlyWage, aMonthlyHours);
        hourlyWage = anHourlyWage;
        monthlyHours = aMonthlyHours;
    }
    // ...
}
\end{lstlisting}


\end{frame}
%------------------------------------------------------------------------

%------------------------------------------------------------------------
\begin{frame}[fragile]{Liskov Substitution Principle (LSP)}
\vspace{-.05in}
\begin{quote}
Subtypes must be substitutable for their supertypes.
\end{quote}
\vspace{-.075in}
A suprising counter-example:
\vspace{-.05in}
\begin{lstlisting}[language=Java]
public class Rectangle {
  public void setWidth(double w) { ... }
  public void setHeight(double h) { ... }
}
public class Square extends Rectangle {
  public void setWidth(double w) {
    super.setWidth(w);
    super.setHeight(w);
  }
  public void setHeight(double h) {
    super.setWidth(h);
    super.setHeight(h);
  }
}
\end{lstlisting}
\vspace{-.1in}
\begin{itemize}
\item We know from math class that a square ``is a'' rectangle. 
\item The overridden {\tt setWidth} and {\tt setHeight} methods in {\tt Square} enforce the class invariant of {\tt Square}, namely, that {\tt width == height}.
\end{itemize}


\end{frame}
%------------------------------------------------------------------------

%------------------------------------------------------------------------
\begin{frame}[fragile]{LSP Violation}


Consider this client of {\tt Rectangle}:
\begin{lstlisting}[language=Java]
public void g(Rectangle r) {
  r.setWidth(5);
  r.setHeight(4);
  assert r.area() == 20;
}
\end{lstlisting}

\begin{itemize}
\item Client (author of {\tt g}) assumes width and height are independent in {\tt r} becuase {\tt r} is a {\tt Rectangle}.
\item If the {\tt r} passed to {\tt g} is actually an instance of {\tt Square}, what will be the value of {\tt r.area()}?
\end{itemize}
The Object-oriented {\tt is-a} relationship is about behavior.  {\tt Square}'s {\tt setWidth} and {\tt setHeight} methods don't behave the way a {\tt Rectangle}'s {\tt setWidth} and {\tt setHeight} methods are expected to behave, so a {\tt Square} doesn't fit the object-oriented {\it is-a} {\tt Rectangle} definition.  Let's make this more formal ...

\end{frame}
%------------------------------------------------------------------------

%------------------------------------------------------------------------
\begin{frame}[fragile]{Conforming to LSP: Design by Contract}


\begin{quote}
Require no more, promise no less.
\end{quote}

Author of a class specifies the behavior of each method in terms of preconditions and postconditions.  Subclasses must follow two rules:
\begin{itemize}
\item Preconditions of overriden methods must be equal to or weaker than those of the superclass (enforces or assumes no more than the constraints of the superclass method).
\item Postconditions of overriden methods must be equal to or greater than those of the superclass (enforces all of the constraints of the superclass method and possibly more).
\end{itemize}

In the Rectangle-Square case the postcondition of {\tt Rectangle}'s {\tt setWidth} method:
\begin{lstlisting}
assert((rectangle.w == w) && (rectangle.height == old.height))
\end{lstlisting}
cannot be satisfied by {\tt Square}, which tells us that a {\tt Square} doesn't satisfy the object-oriented {\it is-a} relationship to {\tt Rectangle}.

\end{frame}
%------------------------------------------------------------------------




% %------------------------------------------------------------------------
% \begin{frame}[fragile]{}


% \begin{lstlisting}[language=Java]

% \end{lstlisting}

% \begin{itemize}
% \item
% \end{itemize}


% \end{frame}
% %------------------------------------------------------------------------


\end{document}
