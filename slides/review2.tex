\documentclass{beamer}

\newcommand{\course}{CS 1331 Introduction to Object Oriented Programming}
\newcommand{\lesson}{Review 2: Arrays Inheritance, and Polymorphism}
\newcommand{\code}{http://www.cc.gatech.edu/~simpkins/teaching/gatech/cs1331/code}

\author[Chris Simpkins] 
{Christopher Simpkins \\\texttt{chris.simpkins@gatech.edu}}
\institute[Georgia Tech] % (optional, but mostly needed)

\date[CS 1331]{}

\subject{\lesson}


% If you have a file called "university-logo-filename.xxx", where xxx
% is a graphic format that can be processed by latex or pdflatex,
% resp., then you can add a logo as follows:

% \pgfdeclareimage[width=0.6in]{coc-logo}{cc_2012_logo}
% \logo{\pgfuseimage{coc-logo}}

\mode<presentation>
{
  \usetheme{Berlin}
  \useoutertheme{infolines}

  % or ...

 \setbeamercovered{transparent}
  % or whatever (possibly just delete it)
}

\usepackage{hyperref}
\usepackage{fancybox}
\usepackage{listings}
\usepackage[abbr]{harvard}

\usepackage[english]{babel}
% or whatever

\usepackage[utf8]{inputenc}
% or whatever

\usepackage{times}
\usepackage[T1]{fontenc}
% Or whatever. Note that the encoding and the font should match. If T1
% does not look nice, try deleting the line with the fontenc.


\usepackage{listings}
 
% "define" Scala
\lstdefinelanguage{scala}{
  morekeywords={abstract,case,catch,class,def,%
    do,else,extends,false,final,finally,%
    for,if,implicit,import,match,mixin,%
    new,null,object,override,package,%
    private,protected,requires,return,sealed,%
    super,this,throw,trait,true,try,%
    type,val,var,while,with,yield},
  otherkeywords={=>,<-,<\%,<:,>:,\#,@},
  sensitive=true,
  morecomment=[l]{//},
  morecomment=[n]{/*}{*/},
  morestring=[b]",
  morestring=[b]',
  morestring=[b]""",
}

\usepackage{color}
\definecolor{dkgreen}{rgb}{0,0.6,0}
\definecolor{gray}{rgb}{0.5,0.5,0.5}
\definecolor{mauve}{rgb}{0.58,0,0.82}
 
% Default settings for code listings
\lstset{frame=tb,
  language=scala,
  aboveskip=2mm,
  belowskip=2mm,
  showstringspaces=false,
  columns=flexible,
  basicstyle={\scriptsize\ttfamily},
  numbers=none,
  numberstyle=\tiny\color{gray},
  keywordstyle=\color{blue},
  commentstyle=\color{dkgreen},
  stringstyle=\color{mauve},
  frame=single,
  breaklines=true,
  breakatwhitespace=true,
  keepspaces=true
  %tabsize=3
}


\title[\course] % (optional, use only with long
                                      % paper titles)
{\lesson}

\subtitle{}
%% {Include Only If Paper Has a Subtitle}

\newcommand{\link}[2]{\href{#1}{\textcolor{blue}{\underline{#2}}}}


% If you wish to uncover everything in a step-wise fashion, uncomment
% the following command: 

% \beamerdefaultoverlayspecification{<+->}


\begin{document}

\begin{frame}
  \titlepage
\end{frame}




%------------------------------------------------------------------------
\begin{frame}[fragile]{Arrays}


Java Arrays (\href{http://docs.oracle.com/javase/specs/jls/se7/html/jls-10.html}{JLS \S 10}):
\begin{itemize}
\item are objects,
\item are dynamically allocated (e.g., with operator {\tt new}),
\item have a fixed number of elements of the same type, and
\item elements are accessed by an index that ranges from 0 to the length of the array - 1.
\end{itemize}



\end{frame}
%------------------------------------------------------------------------

%------------------------------------------------------------------------
\begin{frame}[fragile]{Creating Arrays}


Consider the following array definitions:
\begin{lstlisting}[language=Java]
double[] scores = new double[5];
Dog[] dogs = new Dog[5];
\end{lstlisting}

\begin{itemize}
\item {\tt scores} now contains 5 {\tt double} values, each initialized to 0.0
\item {\tt dogs} now contains 5 reference values of type {\tt Dog}, each initialized to {\tt null}
\item In general each element of the array is initialized to the default value for the element type ({\tt 0} for numeric types, {\tt false} for {\tt boolean} types, and {\tt null} for references \href{http://docs.oracle.com/javase/specs/jls/se7/html/jls-4.html#jls-4.12.5}{JLS \S 4.12.5}).

\end{itemize}

\end{frame}
%------------------------------------------------------------------------

%------------------------------------------------------------------------
\begin{frame}[fragile]{Inheritance and Polymorphism}


Consider
\begin{lstlisting}[language=Java]
public abstract class Animal { public abstract void speak(); }
public class Mammal {
    public void speak() { System.out.println("Hello!"); }
}
public class Dog extends Mammal {
    public void speak() { System.out.println("Woof, woof!"); }
    public void wagTail() { System.out.println("(wags tail)"); }        
}
public class Cat extends Mammal {
    public void speak() { System.out.println("Meow!"); }  
}
\end{lstlisting}

We'll use these classes in the examples in the remaining slides.

\end{frame}
%------------------------------------------------------------------------

%------------------------------------------------------------------------
\begin{frame}[fragile]{Assignments}


Each variable has a static, or compile-time type, and a dynamic, or run-time type.
\begin{itemize}
\item The type of the l-value (to the left of the {\tt =} symbol) in an assignment statement is the static type
\item The type of the r-value (to the right of the {\tt =} symbol) in an assignment statement is the dynamic type
\item In general, for reference assignments the type of the r-value must be a subclass of the type of the l-value
\item Remember that every class is a subclass of itself.
\end{itemize}

So this is fine:
\begin{lstlisting}[language=Java]
Animal fido = new Dog();
\end{lstlisting}
But this is not:
\begin{lstlisting}[language=Java]
Dog spot = new Mammal();  // Error: Mammal not a subclass of Dog
\end{lstlisting}


\end{frame}
%------------------------------------------------------------------------

%------------------------------------------------------------------------
\begin{frame}[fragile]{Casting and Method Binding}


Casting affects static types (some would say ``casting shuts the compiler up'') but method binding is still based on dynamic type.  So
\begin{lstlisting}[language=Java]
Dog fido = new Dog();
((Mammal) fido).speak();
\end{lstlisting}
Produces
\begin{lstlisting}[language=bash]
Woof, woof!
\end{lstlisting}
Even though {\tt Mammal}s say ``Hello!'' because the run-time type of {\tt spot} is still {\tt Dog}.

\end{frame}
%------------------------------------------------------------------------

%------------------------------------------------------------------------
\begin{frame}[fragile]{Upcasting and Downcasting}


The assignment statements we've seen so far are examples of implicit upcasting.
\begin{itemize}
\item Upcasting means treating a reference as an instance of one of its superclasses.
\item Upcasting is safe becuase every object contains the elements of each of its superclasses.
\item Downcasting means treating a reference as an instance of one of its subclasses
\item Downcasting is not safe in general because subclasses may add methods not present in superclasses.  This is why Java doesn't implicitly downcast in assignment statements.
\end{itemize}
Think of upcasting as ``going up'' the class hierarchy and downcasting as ``going down'' the class hierarchy.


\end{frame}
%------------------------------------------------------------------------

%------------------------------------------------------------------------
\begin{frame}[fragile]{Upcasting and Downcasting Examples}


Consider the following:
\begin{lstlisting}[language=Java]
1: Mammal mittens = (Mammal) new Cat(); // Safe
2: Mammal sparky = new Mammal();
3: // Compiles, but will cause a ClassCastException at run-time,
4: Dog huh = (Dog) sparky;
5: // so we won't even get here.
6: huh.wagTail();
\end{lstlisting}

\begin{itemize}
\item The upcast in line 1 is fine.
\item The downcast in line 4 will compile but will cause a ClassCastException at run-time.
\item We won't even get to line 6 due to the exception, which is good because a mammal doesn't have a {\tt wagTail} method.  This is what the ClassCastException is guarding against.
\end{itemize}


\end{frame}
%------------------------------------------------------------------------

% %------------------------------------------------------------------------
% \begin{frame}[fragile]{}


% \begin{lstlisting}[language=Java]

% \end{lstlisting}

% \begin{itemize}
% \item
% \end{itemize}


% \end{frame}
% %------------------------------------------------------------------------


\end{document}
