\documentclass{beamer}

\newcommand{\course}{CS 1331 Introduction to Object Oriented Programming}
\newcommand{\lesson}{Conditional Execution}
\newcommand{\code}{http://www.cc.gatech.edu/~simpkins/teaching/gatech/cs1331/code}

\author[Chris Simpkins] 
{Christopher Simpkins \\\texttt{chris.simpkins@gatech.edu}}
\institute[Georgia Tech] % (optional, but mostly needed)

\date{}

\subject{\lesson}


% If you have a file called "university-logo-filename.xxx", where xxx
% is a graphic format that can be processed by latex or pdflatex,
% resp., then you can add a logo as follows:

% \pgfdeclareimage[width=0.6in]{coc-logo}{cc_2012_logo}
% \logo{\pgfuseimage{coc-logo}}

\mode<presentation>
{
  \usetheme{Berlin}
  \useoutertheme{infolines}

  % or ...

 \setbeamercovered{transparent}
  % or whatever (possibly just delete it)
}

\usepackage{hyperref}
\usepackage{fancybox}
\usepackage{listings}
\usepackage[abbr]{harvard}

\usepackage[english]{babel}
% or whatever

\usepackage[utf8]{inputenc}
% or whatever

\usepackage{times}
\usepackage[T1]{fontenc}
% Or whatever. Note that the encoding and the font should match. If T1
% does not look nice, try deleting the line with the fontenc.


\usepackage{listings}
 
% "define" Scala
\lstdefinelanguage{scala}{
  morekeywords={abstract,case,catch,class,def,%
    do,else,extends,false,final,finally,%
    for,if,implicit,import,match,mixin,%
    new,null,object,override,package,%
    private,protected,requires,return,sealed,%
    super,this,throw,trait,true,try,%
    type,val,var,while,with,yield},
  otherkeywords={=>,<-,<\%,<:,>:,\#,@},
  sensitive=true,
  morecomment=[l]{//},
  morecomment=[n]{/*}{*/},
  morestring=[b]",
  morestring=[b]',
  morestring=[b]""",
}

\usepackage{color}
\definecolor{dkgreen}{rgb}{0,0.6,0}
\definecolor{gray}{rgb}{0.5,0.5,0.5}
\definecolor{mauve}{rgb}{0.58,0,0.82}
 
% Default settings for code listings
\lstset{frame=tb,
  language=scala,
  aboveskip=2mm,
  belowskip=2mm,
  showstringspaces=false,
  columns=flexible,
  basicstyle={\scriptsize\ttfamily},
  numbers=none,
  numberstyle=\tiny\color{gray},
  keywordstyle=\color{blue},
  commentstyle=\color{dkgreen},
  stringstyle=\color{mauve},
  frame=single,
  breaklines=true,
  breakatwhitespace=true,
  keepspaces=true
  %tabsize=3
}


\title[\course] % (optional, use only with long
                                      % paper titles)
{\lesson}

\subtitle{}
%% {Include Only If Paper Has a Subtitle}

\newcommand{\link}[2]{\href{#1}{\textcolor{blue}{\underline{#2}}}}


% \beamerdefaultoverlayspecification{<+->}


\begin{document}

\begin{frame}
  \titlepage
\end{frame}

%------------------------------------------------------------------------
\begin{frame}[fragile]{Boolean Values}


There are 10 kinds of people:
\pause
\begin{itemize}
\item Those who know binary,
\pause
\item and those who don't.
\end{itemize}


\end{frame}
%------------------------------------------------------------------------

%------------------------------------------------------------------------
\begin{frame}[fragile]{Boolean Values}


In Java, boolean values have the {\tt boolean} type.  Four kinds of boolean expressions:
\begin{itemize}
\item {\tt boolean} literals: {\tt true} and {\tt false}
\item {\tt boolean} variables
\item expressions formed by combining non-{\tt boolean} expressions with comparison operators
\item expressions formed by combining {\tt boolean} expressions with logical operators
\end{itemize}


\end{frame}
%------------------------------------------------------------------------


%------------------------------------------------------------------------
\begin{frame}[fragile]{Boolean Expressions}


Simple boolean expressions formed with comparison operators:
\begin{itemize}
\item Equal to: {\tt ==}, like $=$ in math
  \begin{itemize}
    \item Remember, {\tt =} is assignment operator, {\tt ==} is comparison operator!
  \end{itemize}
\item Not equal to: {\tt !=}, like $\ne$ in math
\item Greater than: {\tt >}, like $>$ in math
\item Greater than or equal to: {\tt >=}, like $\ge$ in math
\item ...
\end{itemize}
Examples:
\begin{lstlisting}[language=Java]
1 == 1 // true
1 != 1 // false
1 >= 1 // true
1 > 1  // false
\end{lstlisting}

\end{frame}
%------------------------------------------------------------------------

%------------------------------------------------------------------------
\begin{frame}[fragile]{Combining Boolean Expressions}


Simple boolean expressions can be combined to form larger expressions using:
\begin{itemize}
\item And: {\tt \&\&}, like $\land$ in math
\item Or: {\tt ||}, like $\lor$ in math
\end{itemize}
Examples:
\vspace{-.05in}
\begin{lstlisting}[language=Java]
if ((kids !=0) && ((pieces / kids) >= 2))
    System.out.println("Each kid may have two pieces.");
\end{lstlisting}
\vspace{-.05in}
In this example, note that Java uses short-circuit evaluation.  If
{\tt kids !=0} evaluates to {\tt false}, then the second sub-expression is not evaluated, thus avoiding a divide-by-zero error.

\vspace{.1in}
Note: You can force a complete evaluation by using {\tt \&} or {\tt |}, for example if you have side effects you want to ensure happen in the second expression.  We mention this fact for completeness but implore you not to write such code.\\

Play with \link{\code/Conditionals.java}{Conditionals.java}.

\end{frame}
%------------------------------------------------------------------------

%------------------------------------------------------------------------
\begin{frame}[fragile]{Structured Programming}


Control flow issues:
\begin{itemize}
\item Multiple vs. single entry
\item Multiple vs. single exit
\item {\tt goto} considered harmful
\end{itemize}

Structured programming: block structure, single entry, single exit, no {\tt goto}.  All algorithms expressed by:
\begin{itemize}
\item Sequence - one statement after another
\item Selection - conditional execution (not conditional jumping)
\item Iteration - loops
\end{itemize}

Today we'll learn Java's support for conditoinal execution\footnote{I
  don't like the book's term ``branching.''  Branching is not
  structured programming.  Branching is how you write ``spaghetti code.''}.

\end{frame}
%------------------------------------------------------------------------

%------------------------------------------------------------------------
\begin{frame}[fragile]{The {\tt if-else} Statement}


Conditional execution:
\begin{lstlisting}[language=Java]
if (booleanExpression)
    // a single statement executed when {\tt true}
else
    // a single statement executed when {\tt false}
\end{lstlisting}
\vspace{-.1in}
\begin{itemize}
\item {\tt booleanExpression} must be enclosed in parentheses
\item {\tt else} not required
\end{itemize}
\vspace{-.1in}
Example:
\begin{lstlisting}[language=Java]
if ((num % 2) == 0)
    System.out.printf("%d is even.%n", num);
else
    System.out.printf("%d is odd.%n", num);
\end{lstlisting}

\end{frame}
%------------------------------------------------------------------------

%------------------------------------------------------------------------
\begin{frame}[fragile]{Ternary If-Else Expression}


The ordinary {\tt if-else} control structure is a statement, leading
to conditional assignment code like this: 
\begin{lstlisting}[language=Java]
String dinner = null;
if (temp > 60) {
    dinner = "grilled";
} else {
    dinner = "baked";
}
\end{lstlisting}

The ternary operator combines the above into one expression (recall
that expressions have values):

\begin{lstlisting}[language=Java]
String dinner = (temp > 60) ? "grilled" : "baked" 
\end{lstlisting}

\end{frame}
%------------------------------------------------------------------------


%------------------------------------------------------------------------
\begin{frame}[fragile]{Blocks}


Java is block-structured.  You can enclose any number
of statements in curly braces (\{ ... \}) to create a block.  Blocks
are like single statements (not expressions - they don't have values).
\begin{lstlisting}[language=Java]
if ((num % 2) == 0) {
    System.out.printf("%d is even.%n", num);
    System.out.println("I like even numbers.");
} else {
    System.out.printf("%d is odd.%n", num);
    System.out.println("I'm ambivalent about odd numbers.");
}
\end{lstlisting}
The Java conventions recommend using braces always, even for single
statements.  A very common error is adding statements to an if-branch
and forgetting to add braces.

\end{frame}
%------------------------------------------------------------------------

%------------------------------------------------------------------------
\begin{frame}[fragile]{Multiway {\tt if-else} Statements}


This is hard to folllow:
\begin{lstlisting}[language=Java]
if (color.toUpperCase().equals("RED")) {
    System.out.println("Redrum!");
} else {
    if (color.toLowerCase().equals("yellow")) {
        System.out.println("Submarine");
    } else {
        System.out.println("I got nuthin.");
    }
}
\end{lstlisting}
\vspace{-.05in}
This multiway {\tt if-else} is equivalent, and clearer:
\vspace{-.05in}
\begin{lstlisting}[language=Java]
if (color.toUpperCase().equals("RED")) {
    System.out.println("Redrum!");
} else if (color.toLowerCase().equals("yellow")) {
    System.out.println("Submarine");
} else {
    System.out.println("I got nuthin.");
}
\end{lstlisting}

\end{frame}
%------------------------------------------------------------------------

%------------------------------------------------------------------------
\begin{frame}[fragile]{The {\tt switch} Statement}

\vspace{-.1in}
Java provides {\tt switch} statement for multi-way branching.
\vspace{-.05in}
\begin{lstlisting}[language=Java]
switch (expr) {
    case 1:
        // executed only when case 1 holds
        break;
    case 2:
        // executed only when case 2 holds
    case 3:
        // executed whenever case 2 and 3 hold
        break;
    default:
        // executed only when other cases don't hold        
}
\end{lstlisting}
\vspace{-.1in}
\begin{itemize}
\item Type of {\tt expr} can be {\tt char}, {\tt int}, {\tt short}, {\tt byte}, or {\tt String}
\item In example above, what is type of {\tt expr}?
\item {\tt switch} considered harmful.  97\% of fall-throughs
  unwanted\footnote{Peter van der Linden, {\it Deep C Secrets}}
\item Anachronism from ``structured assembly language'', a.k.a. C (a {\tt switch} is just a jump table)
\end{itemize}

Play with \link{\code/Switch.java}{Switch.java}

\end{frame}
%------------------------------------------------------------------------

%------------------------------------------------------------------------
\begin{frame}[fragile]{Precedence and Associativity}


If an expression contains no parentheses, Java evaluates using precedence and associativity rules (Page 126 in your textbook) in a three-step process:
\begin{enumerate}
\item Associate operands with operators, starting with highest-precedence operators.  This step effectively parenthesizes expression (book calls this ``binding'')
\item Evaluate subexpressions in left to right order (possibly in multiple sweeps if deeply nested)
\item Evaluate outer ``top-level'' operation once all subexpressions have been evaluated
\end{enumerate}


\end{frame}
%------------------------------------------------------------------------

%------------------------------------------------------------------------
\begin{frame}[fragile]{Evaluation Examples}


The expression {\tt 6 + 7 * 2 - 12} is evaluated in the following steps:
\begin{lstlisting}[language=Java]
((6 + (7 * 2)) - 12) // Associate operands with operators
((6 + 14) - 12)      // Evaluate subexpressions ...
(20 - 12)
8
\end{lstlisting}
\vspace{-.05in}
Beware of side-effects.  Consider the evaluation of {\tt ((result = (++n)) + (other = (2*(++n))))} for {\tt n = 2}:
\begin{lstlisting}[language=Java]
((result = (++n)) + (other = (2*(++n))))
((result = 3) + (other = (2*(++n))))
(3 + (other = (2*(++n))))
(3 + (other = (2*4))) // Note that n was 3 from the first pre-increment
(3 + (other = 8))
(3 + 8)
11
\end{lstlisting}
\vspace{-.05in}
Three side-effects: result = 3, other = 8, and n = 4

\end{frame}
%------------------------------------------------------------------------


%------------------------------------------------------------------------
\begin{frame}[fragile]{Closing Thoughts}


\begin{itemize}
\item Conditional execution straigtforward, but watch out for side-effects in boolean assignments.
\item Parenthesize your expressions to make them clear to the reader and to Java.
\item Next we'll learn loops, and we'll have all the tools we need to implement any algorithm.
\end{itemize}


\end{frame}
%------------------------------------------------------------------------


\end{document}
